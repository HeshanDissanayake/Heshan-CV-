%%%%%%%%%%%%%%%%%%%%%%%%%%%%%%%%%%%%%%%%%
% Medium Length Professional CV
% LaTeX Template
% Version 2.0 (8/5/13)
%
% This template has been downloaded from:
% http://www.LaTeXTemplates.com
%
% Original author:
% Trey Hunner (http://www.treyhunner.com/)
%
% Important note:
% This template requires the resume.cls file to be in the same directory as the
% .tex file. The resume.cls file provides the resume style used for structuring the
% document.
%
%%%%%%%%%%%%%%%%%%%%%%%%%%%%%%%%%%%%%%%%%

%----------------------------------------------------------------------------------------
%	PACKAGES AND OTHER DOCUMENT CONFIGURATIONS
%----------------------------------------------------------------------------------------

\documentclass{resume} % Use the custom resume.cls style
\usepackage{hyperref}
\usepackage{graphicx}

\usepackage[left=0.75in,top=0.6in,right=0.75in,bottom=0.6in]{geometry} % Document margins
\newcommand{\tab}[1]{\hspace{.2667\textwidth}\rlap{#1}}
\newcommand{\itab}[1]{\hspace{0em}\rlap{#1}}



\clearpage
\name{Heshan Dissanayake} % Your name
\address{{Department of Computer Engineering
      Faculty of Engineering
      University of Peradeniya
      Sri Lanka}} % Your address
%\address{123 Pleasant Lane \\ City, State 12345} % Your secondary addess (optional)
\address{\href{mailto:e16088@eng.pdn.ac.lk}{e16088@eng.pdn.ac.lk} \\ +94750365739 \\
\href{https://heshandissanayake.github.io/}{Portfolio}}
\begin{document}

%----------------------------------------------------------------------------------------
%	profile
%----------------------------------------------------------------------------------------

\begin{rSection}{Profile}
\begin{rSubsection}{}{}{}{}
A computer Engineering undergraduate with a motivation to explore the underlying concepts and theories to invent creative solutions to existing problems. I have interests in vision, robotics and the machine learning application in the field of robotics. I have done several interesting projects to explore the real application aspect of my interests. Perform well in my academic and self initiated projects is something that I consider as important. I believe that achieving the solution for a certain problem requires leadership, management, and communication skills, qualities which I develop every day in my life.


\end{rSubsection}


\end{rSection}


%----------------------------------------------------------------------------------------
%	Interests
%----------------------------------------------------------------------------------------

\begin{rSection}{Interests}


\begin{tabular}{ @{} >{\hspace{6ex}}l @{\hspace{6ex}} l @{\hspace{6ex}}l}
Robotics and Automation & Computer Vision & Embedded Systems\\
Algorithmic Programming & Machine Learning\\
\end{tabular}
\end{rSection}


%----------------------------------------------------------------------------------------
%	EDUCATION SECTION
%----------------------------------------------------------------------------------------
\begin{rSection}{Education}

{\bf \href{http://eng.pdn.ac.lk}{University of Peradeniya}} \hfill {\em 2017 Nov - Present} 
\\ \href{http://eng.pdn.ac.lk/deee/academic/undergraduate.php}{Undergraduate in BSc. Engineering(Hons.)} \hfill {\bf  GPA: 3.6/4.00}
%Minor in Linguistics \smallskip \\
%Member of Eta Kappa Nu \\
%Member of Upsilon Pi Epsilon \\

{\bf{Kingswood College ,Kandy}} \hfill {\em 2003 - 2016} 
\\ G.C.E. Advanced Level Examination
\\ District Rank - 108, Island Rank - 1200 \hfill {\bf  Z-Score: 1.83}
\\ Physics (A), Chemistry (A), Combined Mathematics (B)
\\
\\ G.C.E. Ordinary Level Examination
\\ A passes for all 9 subjects

% \begin{rSubsection}{}{}{}{}
% \end{rSubsection}
\end{rSection}



%----------------------------------------------------------------------------------------
%	TECHNICAL STRENGTHS SECTION
%----------------------------------------------------------------------------------------
% \clearpage

\begin{rSection}{Skills}

\begin{tabular}{ @{} >{\bfseries}l @{\hspace{6ex}} l }
Programming Languages &  Python, Java, JavaScript, C, C++ \\
Numerical Computing Packages &  MATLAB, Octave, Numpy, TensorFlow \\
Procedural programming & ARM Assembly \\
Hardware Programming  & AVR programming, Verilog HDL \\
PCB Designing & Eagle, Altim \\
3D Modelling & AutoCAD, Fusion360\\
Version control & git \\
Practical Skills  & Soldering, PCB design and development\\
Languages &  English, Sinhala \\

\end{tabular}
\end{rSection}


\clearpage
%----------------------------------------------------------------------------------------

\begin{rSection}{Projects}\\

\textbf{Group Projects}\\

\begin{rSubsection}{\href{https://github.com/dtdinidu7/e16-3yp-obstacle-bots-for-swarm-robots}{Obstacle robot swarm for swarm robotic project}}{2020-2021}{}{}
\item A system of obstacle robots for a swarm robotic platform. \item 
\textit{Technologies: Python, OpenCV, numpy, MQTT, JavaScript, GRPC}
\item \textit{Techniques: Image Processing, stochastic gradient descent, Encryption }
\end{rSubsection}

\begin{rSubsection}{\href{https://github.com/HeshanDissanayake/8-bit-computer}{8-bit Computer}}{2020}{}{}
\item Design and building a 8-bit computer. \item 
\textit{Technologies: Embedded system, Integrated circuits}
\item \textit{Techniques: Computer Architecture }
\end{rSubsection}

\begin{rSubsection}{Micromouse}{2019}{}{}
\item Autonomous maze navigation robot using custom made sensors 
\item \textit{Technologies: Arduino Microcontroller, IR Sensors, Gyroscope} 
\item \textit{Techniques: Graph Theory, PID Control Systems, Sensor Calibration}
\end{rSubsection}

\begin{rSubsection}{\href{https://github.com/HeshanDissanayake/SIIM_ISIC_melanoma_classification}{SIIM-ISIC Melanoma Classification}}{2020}{}{}
\item Identify melanoma in lesion images. \item 
\textit{Technologies: Python, Tensorflow, numpy}
\item \textit{Techniques: Image Processing, Convolution Neural Networks, Transfer Learning }
\newline

\begin{rSubsection}{Intelligent CCTV System}{2019}{}{}
\item Tracking people and unattended baggage using a neural network based CCTV System. \item 
\textit{Technologies: Python, Numpy, OpenCV, TensorFlow}
\item \textit{Techniques: Neural Networks, Data Clustering}
\end{rSubsection}

\begin{rSubsection}{Aerial Sensoring using Hyperspectral Imagery for Soil Moisture Detection }{2018}{}{}
\item Using Hyperspectral images taken from satellites and drones to estimate soil moisture content. 
\item \textit{Technologies: Python, Numpy, TensorFlow}
\item \textit{Techniques: Hyperspectral Data manipulation, Neural Networks}
\end{rSubsection}

\begin{rSubsection}{Ambulatory Wound Monitor }{2018}{}{}
\item A small portable sensor that can be embedded in wounds to monitor parameter such as temperature, pH and dressing pressure, in order to monitor the health of wounds
\item \textit{Technologies: Arduino Micro controller, Bluetooth Communication}
\end{rSubsection}


\begin{rSubsection}{Analog line Follower Robot}{2018}{}{}
\item Analog Line Follower (PD Controller based)
\item \textit{Technologies: Op Amps}
\item \textit{Techniques: PD controlling}
\end{rSubsection}

\begin{rSubsection}{Landslide Detection System}{2018}{}{}
\item A prototype device which monitors shear strain of soil in landslide prone areas in order to predict landslides.
\item \textit{Technologies: Arduino Micro controller, WiFi Communication}
\end{rSubsection}


\end{rSubsection}

\textbf{Individual Projects}\\

\begin{rSubsection}{\href{https://github.com/HeshanDissanayake/Convolutional_Auto_encoder}{Convolution Auto Encoder for Person Re-identification}}{2020}{}{}
\item Using Auto Encodes for Convolution neural networks to identify a predefined person. \item 
\textit{Technologies: Python, Tensorflow, numpy}
\item \textit{Techniques: Image Processing, Auto encoders, Convolution Neural Networks}
\end{rSubsection}

\begin{rSubsection}{Bird Watcher system}{2020-2021}{}{}
\item A system to watch birds from remote streaming devices \item 
\textit{Technologies: Python, RTMP, OpenCV, MQTT, JavaScript, ffmepg, nginx, Flutter, Google Vision AI}
\item \textit{Techniques: Real time video streaming, Motion Detection}
\end{rSubsection}


\begin{rSubsection}{Verilog Based CPU}{2020}{}{}
\item Designing of a 32-bit CPU which supports simple instructions with caching. \item 
\textit{Technologies: Verilog}
\item \textit{Techniques: Computer Architecture }
\end{rSubsection}


\end{rSection}

%----------------------------------------------------------------------------------------

%	ACHIEVEMENTS
%----------------------------------------------------------------------------------------

\begin{rSection}{Achievements}

{\bf DataStorm 1.0} \hfill {\em 2020}
\\2nd Runners up
\\Task : Credit Card Default Prediction


{\bf ACES Hackathon} \hfill {\em 2019}
\\1st place in Travel and Safety Category
\\Project : Neural Network based CCTV System for tracking individuals and unattended baggage

{\bf SLIIT Robofest} \hfill {\em 2019}
\\3rd place in the undergraduate category
\\Task : Autonomous Maze Navigating Robot (Micromouse)

{\bf ACES Hackathon} \hfill {\em 2018}
\\1st place in Network and System Category
\\Project : Landslide Detection System

{\bf Selected to Faculty of Engineering, University of Peradeniya} \hfill {\em 2016}
\\District Rank - 107,  Island Rank - 1200
\\Z - score - 1.83

{\bf 9A passes in GCE Ordinary Level} \hfill {\em 2013} 
\end{rSection}


% \clearpage
%----------------------------------------------------------------------------------------
\begin{rSection}{Extra-curricular} \itemsep -3pt
\item Committee member of the Hacker's club of the University of Peradeniya (2020 - Present)
\item Member of the Music Society of the University of Peradeniya (2018 - Present)
\item Committee member of Astronomy Club of KingsWood College Kandy (2016)
\item Member of Science Society of KingsWood College Kandy (2016)
\item Member of Photography of KingsWood College Kandy (2016)

 
\end{rSection}

\clearpage
\begin{rSection}{Other Interests and Hobbies} \itemsep -3pt
\item 3D modeling and digital art Enthusiast.
\item Drawing and Painting Enthusiast.
\item Amature Astronomer.


\end{rSection}

% REFERENCES

\begin{rSection}{References}
\itab{\href{}{\textbf{Prof. Roshan G. Ragel} }} \\
% \itab{BSc Eng} \\ 
\itab{Professor, Dept. of Computer Engineering} \\
\itab{Univeristy of Peradeniya} \\  
\itab{\href{mailto:roshanr@eng.pdn.ac.lk}{roshanr@eng.pdn.ac.lk}}\\
%\itab{+94-77-3857755} \\

\itab{\href{}{\textbf{Dr. Isuru Nawinne} }} \\
% \itab{BSc Eng} \\ 
\itab{Senior Lecturer, Dept. of Computer Engineering} \\
\itab{Univeristy of Peradeniya} \\  
\itab{\href{mailto:isurunawinne@eng.pdn.ac.lk}{isurunawinne@eng.pdn.ac.lk}}\\
%\itab{+94-71-8495506} \\

\end{rSection}

\end{document}
